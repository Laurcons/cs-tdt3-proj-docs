\documentclass{article}
\usepackage{graphicx} % Required for inserting images
\usepackage[unicode]{hyperref}
\usepackage[onehalfspacing]{setspace}           % 1.5 distance between lines
\usepackage{palatino}                           % font

\bibliographystyle{plain}

\title{TDT Project Documentation}
\author{Stanciu Alin, Pricop Laurentiu}
\date{Academic Year 2023-2024, Spring Semester}

\begin{document}
\pagenumbering{roman}

\maketitle
\tableofcontents

\newpage
\pagenumbering{arabic}

\section{Application Details. Investigated features}

The application we are testing is called Physical Mail Manager. The application is available for fiddling around with publicly \cite{PmailSite}, and its code is available on GitHub \cite{PmailGithub}. We will refer to the app as "PMail" from now on.

PMail intends to provide a means of keeping track of physical mail (alternatively known as snail mail) that Laurentiu personally was sending and receiving. It allows configuring destinations and can assign to each letter a unique trackable code.

PMail provides the following features:
\begin{enumerate}
    \item Authentication based on username and password, along with registration.
    \item Maintain a list of receivers, along with information required on the envelope (address, full name, postal code) and other optional user-defined information.
    \item Store a list of letters. Each letter is attached to a receiver (which keeps the same name even if they are actually a sender), can store metadata such as letter unique code, price, and other optional user-defined information.
\end{enumerate}

\section{AC. IOs}

\section{Testing Mission}

\section{Testing Strategy}

\section{Selected Test Design Techniques}

\section{Test Design. Test implementation. Test execution. Test Report}

\section{Issue Reporting}

\section{Conclusions. Lessons Learned}

\newpage
\bibliography{references}

\end{document}
